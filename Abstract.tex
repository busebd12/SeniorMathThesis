\documentclass[executivepaper]{article}

\usepackage{mathtools}

\everymath{\displaystyle}

\usepackage{amssymb}

\usepackage{commath}

\usepackage{kantlipsum,graphicx}

\usepackage{amsmath}

\usepackage{pgfplots}

\usepackage[utf8]{inputenc}

\usepackage{sectsty}

\usepackage{float}

\sectionfont{\large}

\subsectionfont{\normalsize}

\begin{document}

\vspace*{-40mm}

\renewcommand\thesubsection{\thesection.\arabic{subsection}}

\renewcommand{\ttdefault}{cmtt}

\newtheorem{observation}{Observation}

\begin{center}

\section*{Abstract}

The focus of the research presented in this paper was, for the singular boundary value problem given as 

\begin{center}

\vspace{-2mm}

\begin{equation}
\begin{array}{l}
\displaystyle (-x^ny')'=\lambda x^my\\[2ex]
\displaystyle y(0)=0, y(1)=0 \\
\end{array} 
\label{eq:BVP}
\end{equation}

\end{center}

defined on the interval $(0,1)$, to examine the affects of changing the value of $m$ and $n$ on the eigenvalues and eigenfunctions that were obtained. The researcher employed the shooting method by numerically solving the initial value problem $(-x^ny')'=\lambda x^my$, defined on the interval $(0,1)$, and subject to the initial conditions $y'(1)=-1, y(1)=0$ for various values of $\lambda$.

\end{center}

\end{document}